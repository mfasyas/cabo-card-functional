\documentclass{beamer}
\usetheme{ALUF}

\usepackage[utf8]{inputenc}
\usepackage{palatino}
\usepackage[T1]{fontenc}
\usepackage{lmodern}
\usepackage[expert]{mathdesign}
\usepackage[protrusion=true,expansion=true,tracking=true,kerning=true]{microtype}
\usepackage{amsmath, mathrsfs, amssymb}
\usepackage{tikz}

\usepackage{xcolor} % For colors

\usepackage{listings}

\lstdefinestyle{haskellstyle}{
  language=Haskell,
  basicstyle=\ttfamily\tiny,
  keywordstyle=\color{blue},
  commentstyle=\color{green!60!black},
  stringstyle=\color{red},
  numberstyle=\tiny\color{gray},
  numbers=none,
  showstringspaces=false,
  tabsize=2,
  breaklines=true,
  frame=single,
  framerule=0.5pt,
  backgroundcolor=\color{gray!5},
  literate={\$}{{\$}}1, % <--- penting! agar tanda $ tidak bikin error
}


\newcommand\restr[2]{{% we make the whole thing an ordinary symbol
  \left.\kern-\nulldelimiterspace % automatically resize the bar with \right
  #1 % the function
  \littletaller % pretend it's a little taller at normal size
  \right|_{#2} % this is the delimiter
  }}

\newcommand{\littletaller}{\mathchoice{\vphantom{\big|}}{}{}{}}


\title{Permainan Kartu Kabul}
\subtitle{Proyek Pemrograman Fungsional}
\author{Muhammad Fasya Syaifullah}
\date{3 November 2025}
\institute{Universitas Indonesia}

\begin{document}

\begin{frame}[plain,t]
\titlepage
\end{frame}

\begin{frame}{Daftar Isi}
    \tableofcontents
\end{frame}

%=============================================================================================
\section{Kemajuan}
\begin{frame}{Target Sejauh Ini}
    \begin{itemize}
        \item \textbf{Target Progress 1 (27 Oktober 2025)}
        \begin{itemize}
            \item Membuat rancangan Game secara umum.
            \item Menyiapkan stuktur tipe data untuk Kartu, Pemain, Meja dan integrasinya seperti ambil kartu dan buang kartu. 
            \item Tampilan output sederhana untuk masing-masing tipe data.
        \end{itemize}

        \item \textbf{Target Progress 2 (3 November 2025)}
        \begin{itemize}
            \item Operasi-operasi sederhana untuk permainan.
            \item Kartu terambil terbuka, kartu di tangan tertutup.
            \item Modifikasi urutan kartu untuk penempatan strategis.
        \end{itemize}
    \end{itemize}
\end{frame}

\begin{frame}{Pekan 1}
    \begin{figure}
        \centering
        \includegraphics[width=0.65\linewidth]{Showing Table.PNG}
        \caption{\textit{Starting Point}}
        \label{fig:starting point}
    \end{figure}
    \begin{figure}
        \centering
        \includegraphics[width=0.7\linewidth]{Players Turn.PNG}
        \caption{\textit{Player's Turn}}
        \label{fig:starting point}
    \end{figure}
\end{frame}
\begin{frame}[fragile]{Pekan 1}
    \begin{lstlisting}[style=haskellstyle]
        data Card = Card { rank :: Rank, suit :: Suit, powerup :: Powerup} 
            deriving (Eq)
    \end{lstlisting}
    \begin{lstlisting}[style=haskellstyle]
        data Player = Player {playerId :: Int, hand :: Hand} 
            deriving (Show, Eq)
    \end{lstlisting}
    \begin{lstlisting}[style=haskellstyle]
        type Deck = [Card]
        type Pile = [Card]
    \end{lstlisting}
    \begin{lstlisting}[style=haskellstyle]
        data Table = Table{ 
            players :: [Player], drawDeck :: Deck, discardPile :: Pile
        }
    \end{lstlisting}
\end{frame}

\begin{frame}{Pekan 2}
    \begin{figure}
        \centering
        \includegraphics[width=0.8\linewidth]{Activity.PNG}
        \caption{Peek Card}
        \label{fig:peek card}
    \end{figure}
\end{frame}

\begin{frame}{Pekan 2}
    \begin{figure}
        \centering
        \includegraphics[width=0.8\linewidth]{Players Turn2.PNG}
        \caption{Peek Card}
        \label{fig:throwcard}
    \end{figure}
\end{frame}

\begin{frame}[fragile]{Pekan 2}
    \begin{lstlisting}[style=haskellstyle]
        drawForHand :: Table -> Int -> Table
        drawForHand table playerIdx =
            let ps = players table
                deck = drawDeck table
                (drawnCard, newDeck) = drawCard deck
                (before, p:after) = splitAt playerIdx ps
                Hand hs = hand p
                newHand = drawnCard : hs
                newPlayer = p { hand = Hand newHand }
                newPlayers = before ++ [newPlayer] ++ after
            in table { players = newPlayers, drawDeck = newDeck }
    \end{lstlisting}
    \begin{lstlisting}[style=haskellstyle]
        discardFromHand :: Table -> Int -> Int -> Table
        discardFromHand table playerIdx discardIdx =
            let ps = players table
                pile = discardPile table
                (before, p:after) = splitAt playerIdx ps
                Hand hs = hand p
                (toDiscard, keptHand) = removeAt (discardIdx - 1) hs
                newPlayer = p { hand = Hand keptHand }
                newPlayers = before ++ [newPlayer] ++ after
                newPile = toDiscard : pile
            in table { players = newPlayers, discardPile = newPile }
    \end{lstlisting}

\end{frame}

\section{Daftar Commit}
\begin{frame}{Commit}
    \begin{figure}
        \centering
        \includegraphics[width=0.9\linewidth]{log commit.PNG}
        \label{fig:log commit}
    \end{figure}
    \url{https://github.com/mfasyas/cabo-card-functional/commits/main/}
\end{frame}

\section{Hal yang Dipelajari}
\begin{frame}{Aspek Fungsional}
    \begin{itemize}
        \item \textbf{Recursion} untuk state permainan. Setiap kali pemain melakukan aksi, akan diciptakan \textit{state} baru yang merepresentasikan kegiatan pemain di meja.
        \item \textbf{Pure Functions} digunakan pada perhitungan skor dan pengambilan kartu. Hal ini keharusan karena jika suatu kartu tidak pasti, akan berpengaruh ke permainan.
        \item \textbf{Types} pada pendefinisian \textit{Card}, \textit{Player}, dan \textit{Table}. Penggunaan \textit{type} untuk mendefinisikan ketiga hal tersebut adalah untuk merepresentasikan \textit{state} dari permainan yang akan ditunjukkan ke masing-masing pemain.
    \end{itemize}
\end{frame}




%=============================================================================================
\end{document}