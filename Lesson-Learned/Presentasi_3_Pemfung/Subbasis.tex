\documentclass{beamer}
\usetheme{ALUF}

\usepackage[utf8]{inputenc}
\usepackage{palatino}
\usepackage[T1]{fontenc}
\usepackage{lmodern}
\usepackage[expert]{mathdesign}
\usepackage[protrusion=true,expansion=true,tracking=true,kerning=true]{microtype}
\usepackage{amsmath, mathrsfs, amssymb}
\usepackage{tikz}

\usepackage{xcolor} % For colors

\usepackage{listings}

\lstdefinestyle{haskellstyle}{
  language=Haskell,
  basicstyle=\ttfamily\tiny,
  keywordstyle=\color{blue},
  commentstyle=\color{green!60!black},
  stringstyle=\color{red},
  numberstyle=\tiny\color{gray},
  numbers=none,
  showstringspaces=false,
  tabsize=2,
  breaklines=true,
  frame=single,
  framerule=0.5pt,
  backgroundcolor=\color{gray!5},
  literate={\$}{{\$}}1, % <--- penting! agar tanda $ tidak bikin error
}


\newcommand\restr[2]{{% we make the whole thing an ordinary symbol
  \left.\kern-\nulldelimiterspace % automatically resize the bar with \right
  #1 % the function
  \littletaller % pretend it's a little taller at normal size
  \right|_{#2} % this is the delimiter
  }}

\newcommand{\littletaller}{\mathchoice{\vphantom{\big|}}{}{}{}}


\title{Permainan Kartu Kabul}
\subtitle{Proyek Pemrograman Fungsional}
\author{Muhammad Fasya Syaifullah}
\date{11 November 2025}
\institute{Universitas Indonesia}

\begin{document}

\begin{frame}[plain,t]
\titlepage
\end{frame}

\begin{frame}{Daftar Isi}
    \tableofcontents
\end{frame}

%=============================================================================================
\section{Kemajuan}
\begin{frame}{Target Sejauh Ini}
    \textbf{Target Progress 3 (27 Oktober 2025)}
    \begin{itemize}
        \item Mengatur \textit{game logic}.
        \item Menentukan pemenang.
        \item Permainan dasar sudah dapat dilakukan.
        \item Menambahkan perhitungan skor
    \end{itemize}
\end{frame}

\begin{frame}{Tampilan Output}
    \centering
    \includegraphics[width=0.9\linewidth]{folder/1.PNG}
\end{frame}

\begin{frame}{Tampilan Output}
    \centering
    \includegraphics[width=0.9\linewidth]{folder/2.PNG}
\end{frame}

\begin{frame}{Tampilan Output}
    \centering
    \includegraphics[width=0.9\linewidth]{folder/3.PNG}
\end{frame}

\begin{frame}{Tampilan Output}
    \centering
    \includegraphics[width=0.9\linewidth]{folder/4.PNG}
\end{frame}

\begin{frame}{Tampilan Output}
    \centering
    \includegraphics[width=0.9\linewidth]{folder/5.PNG}
\end{frame}

\begin{frame}{Tampilan Output}
    \centering
    \includegraphics[width=0.9\linewidth]{folder/6.PNG}
\end{frame}

\begin{frame}{Tampilan Output}
    \centering
    \includegraphics[width=0.7\linewidth]{folder/end.PNG}
\end{frame}


\begin{frame}[fragile]{Pekan 3}
    \begin{lstlisting}[style=haskellstyle]
playerTurn :: Table -> Int -> IO Table
playerTurn table playerIdx = do
    let currentPlayer = players table !! playerIdx
    putStrLn $ "\nPlayer " ++ show (playerId currentPlayer) ++ "'s turn"
    putStrLn "Do you want to end the game now? (y/n)"
    endChoice <- getLine
    if endChoice == "y"
        then do
            putStrLn $ "\nGame Stopped, Kabul!"
            return table { drawDeck = [] } -- Trigger end case di mainLoop
        else do
            let tableAfterDraw = drawForHand table playerIdx
                updatedPlayer = players tableAfterDraw !! playerIdx
                currentHand = hand updatedPlayer

            putStrLn "Current hand, top of Card is newly drawn Card:"
            putStrLn $ showHandWithIndices currentHand

            choice <- getValidChoice (length $ let Hand h = currentHand in h)
            let tableAfterDiscard = discardFromHand tableAfterDraw playerIdx choice
            putStrLn $ "Card " ++ show choice ++ " has been discarded."
            
            addToPile tableAfterDiscard playerIdx
    \end{lstlisting}
\end{frame}

\begin{frame}[fragile]{Pekan 3}
\begin{lstlisting}[style=haskellstyle]
playRounds :: Table -> Int -> IO Table
playRounds table currentPlayerIdx
    | null (drawDeck table) = do
        putStrLn "\nKartu di dek habis! Permainan berakhir!"
        return table
    | otherwise = do
        print table
        let numPlayers = length $ players table
        
        newTable <- playerTurn table currentPlayerIdx
        
        let nextPlayerIdx = (currentPlayerIdx + 1) `mod` numPlayers
        playRounds newTable nextPlayerIdx
\end{lstlisting}
    
\end{frame}

\section{Daftar Commit}
\begin{frame}{Commit}
    \href{https://github.com/mfasyas/cabo-card-functional/commits/add-feature}{link commit}
\end{frame}

\section{Hal yang Dipelajari}
\begin{frame}{Aspek Fungsional}
    \begin{itemize}
        \item Aspek pure function dilanggar setiap permainan dilakukan pada bagian \texttt{deckBuild}.

        \item Fungsi itu tidak pure karena memakai \texttt{randomRIO} dan mutasi array yang menghasilkan keluaran berbeda tiap pemanggilan. Efeknya hasil shuffle tidak deterministik dan harus dijalankan dalam konteks \texttt{IO}.
    \end{itemize}
\end{frame}

\begin{frame}[fragile]{Hal yang Dipelajari}
\textbf{Algoritma Shuffle Deck} untuk mengacak kartu untuk dibagikan ke pemain.
\begin{lstlisting}[style=haskellstyle]
shuffleDeck :: Deck -> IO Deck
shuffleDeck cards = do
    let len = length cards
    arr <- newListArray (0, len - 1) cards :: IO (IOArray Int Card)
    forM [0 .. len - 2] $ \i -> do
        j <- randomRIO (i, len - 1) -- Get a random index
        vi <- readArray arr i        -- Swap elements
        vj <- readArray arr j
        writeArray arr i vj
        writeArray arr j vi
    mapM (readArray arr) [0 .. len - 1]
\end{lstlisting}
    
\end{frame}

%=============================================================================================
\end{document}