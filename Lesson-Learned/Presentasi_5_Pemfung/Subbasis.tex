\documentclass{beamer}
\usetheme{ALUF}

\usepackage[utf8]{inputenc}
\usepackage{palatino}
\usepackage[T1]{fontenc}
\usepackage{lmodern}
\usepackage[expert]{mathdesign}
\usepackage[protrusion=true,expansion=true,tracking=true,kerning=true]{microtype}
\usepackage{amsmath, mathrsfs, amssymb}

\usepackage{xcolor} % For colors

\usepackage{listings}

\lstdefinestyle{haskellstyle}{
  language=Haskell,
  basicstyle=\ttfamily\tiny,
  keywordstyle=\color{blue},
  commentstyle=\color{green!60!black},
  stringstyle=\color{red},
  numberstyle=\tiny\color{gray},
  numbers=none,
  showstringspaces=false,
  tabsize=2,
  breaklines=true,
  frame=single,
  framerule=0.5pt,
  backgroundcolor=\color{gray!5},
  literate={\$}{{\$}}1, % <--- penting! agar tanda $ tidak bikin error
}


\newcommand\restr[2]{{% we make the whole thing an ordinary symbol
  \left.\kern-\nulldelimiterspace % automatically resize the bar with \right
  #1 % the function
  \littletaller % pretend it's a little taller at normal size
  \right|_{#2} % this is the delimiter
  }}

\newcommand{\littletaller}{\mathchoice{\vphantom{\big|}}{}{}{}}


\title{Permainan Kartu Kabul}
\subtitle{Proyek Pemrograman Fungsional}
\author{Muhammad Fasya Syaifullah}
\date{24 November 2025}
\institute{Universitas Indonesia}

\begin{document}

\begin{frame}[plain,t]
\titlepage
\end{frame}

\begin{frame}{Daftar Isi}
    \tableofcontents
\end{frame}

%=============================================================================================
\section{Kemajuan}
\begin{frame}{Target}
    \textbf{Target Progress 5 (24 November 2025)}
    \begin{itemize}
        \item \textit{Refactor} kode untuk \textit{Declarative Rule System}
        \item Menyicil \textit{User-Interface}
    \end{itemize}

    Kemajuan:
    \begin{itemize}
        \item \textit{Declarative Rule System} berhasil diterapkan namun masih ada kekurangan fitur dibandingkan sebelumnya.
        \item \textit{User Interface} secara garis besar sudah terlihat. Target berikutnya adalah agar dapat dilakukan permainan sebagai mana semestinya.
    \end{itemize}
\end{frame}

\begin{frame}{Tampilan Permainan}
    \begin{figure}
        \centering
        \includegraphics[width=0.8\linewidth]{UI_UX.png}
        \caption{Tampilan Permainan}
    \end{figure}
\end{frame}

\begin{frame}{Flow Permainan}
    \begin{figure}
        \centering
        \includegraphics[width=0.75\linewidth]{InitialState.png}
        \caption{Tampilan Awal Permainan}
    \end{figure}
\end{frame}

\begin{frame}{Flow Permainan}
    \begin{figure}
        \centering
        \includegraphics[width=0.75\linewidth]{InitialState.png}
        \caption{Tampilan Awal Permainan}
    \end{figure}
\end{frame}

\begin{frame}{Fase Draw}
    \begin{figure}
        \centering
        \includegraphics[width=0.55\linewidth]{draw.png}
        \caption{Ambil kartu di deck dan pilih tangan yang di buang}
    \end{figure}
\end{frame}

\begin{frame}{\textit{Trigger Powerup}}
    \begin{columns}
        \begin{column}{0.5\textwidth} 
            \begin{figure}
                \centering
                \includegraphics[width=1\linewidth]{trigger1.png}
                \caption{Pilih target}
            \end{figure}
        \end{column}
        \begin{column}{0.5\textwidth}  
            \begin{figure}
                \centering
                \includegraphics[width=1\linewidth]{trigger2.png}
                \caption{Informasi muncul}
            \end{figure}
        \end{column}
\end{columns}
\end{frame}

\section{Daftar Commit}
\begin{frame}{Link Commit}
    \centering
    \href{https://github.com/mfasyas/cabo-card-functional/blob/main/README.md}{Readme klik di sini}\\
    \vspace{2 cm}
    \href{https://github.com/mfasyas/cabo-card-functional/commits/main/}{Commit klik di sini}
\end{frame}

\section{Hal yang Dipelajari}

\begin{frame}{\textit{Flow} menjadi \textit{Constraint}}
    Usahakan dalam membuat game, buat aturan deklaratif dari awal sebisa mungkin.
    \vspace{1 cm}
    
    Pada kode lama, aturan permainan dikendalikan oleh banyak pernyataan bersarang \textit{if-else} yang tersebar di seluruh fungsi yang dipakai dalam \textit{loop} permainan. Dengan sistem deklaratif, logika menjadi lebih terpusat dan mudah dibaca, serta untuk jangka panjang lebih mudah diperbarui.
\end{frame}

\begin{frame}{Komposisi Fungsi}
    \textit{Refactoring} juga menyoroti keunggulan komposisi dibandingkan dengan \textit{nesting} di mana pada kode yang imperatif, menambahkan aturan baru artinya menambahkan lebih banyak pernyataan \textit{if-else}. 
    
    Sedangkan dalam sistem deklaratif aturan-aturan kompleks disusun dari fungsi-fungsi kecil sederhana dan independen. Artinya, struktur merujuk pada merangkai aturan-aturan kecil yang valid, bukan membangun prosedur raksasa yang menangani banyak kondisi.
\end{frame}

\begin{frame}[fragile]{Contoh 1}
Sebelumnya:
\begin{lstlisting}[style=haskellstyle]
playerTurn table idx = do
    putStrLn "Masukkan kartu:"
    input <- getLine
    let choice = read input
    if choice < 0 || choice >= length hand 
       then do 
             putStrLn "Index salah!"
             playerTurn table idx   -- Rekursif manual
       else ...                     -- Lanjut logic
\end{lstlisting}

Setelahnya:
\begin{lstlisting}[style=haskellstyle]
isPlayerTurn :: Rule
    ...

isPhaseCorrect :: Rule
    ...

isValidIndex :: Rule
isValidIndex gs (DiscardAction _ idx) = checkIndex gs idx

gameRules :: Rule
gameRules = isPlayerTurn .&&. isPhaseCorrect .&&. isValidIndex
\end{lstlisting}
\end{frame}

\begin{frame}[fragile]{Contoh 2 (1)}

\begin{lstlisting}[style=haskellstyle]
data Table = Table
    { players :: [Player]
    , drawDeck :: Deck
    , discardPile :: Deck
    , standings :: [(Int, Int)] -- (idPlayer, score)
    } deriving (Eq)
\end{lstlisting}

\end{frame}

\begin{frame}[fragile]{Contoh 2 (2)}

\begin{lstlisting}[style=haskellstyle]
data GamePhase
    = DrawPhase
    | DiscardPhase
    | ResolvePowerup Powerup
    | GameOver
    deriving (Show, Eq, Generic, ToJSON, FromJSON)

data GameAction
    = DrawAction Int
    | DiscardAction Int Int
    | TargetAction Int (Int, Int) 
    | FinishGameAction Int
    deriving (Show, Eq, Generic, ToJSON, FromJSON)

data GameState = GameState {
        players         :: [Player]
    ,   drawDeck        :: Deck
    ,   discardPile     :: Deck
    ,   currentTurn     :: Int
    ,   phase           :: GamePhase
    ,   logs            :: [String]
    ,   privateInfo     :: [(Int, String)]
    } deriving (Show, Generic, ToJSON, FromJSON)
\end{lstlisting}

\end{frame}


\end{document}