\documentclass{beamer}
\usetheme{ALUF}

\usepackage[utf8]{inputenc}
\usepackage{palatino}
\usepackage[T1]{fontenc}
\usepackage{lmodern}
\usepackage[expert]{mathdesign}
\usepackage[protrusion=true,expansion=true,tracking=true,kerning=true]{microtype}
\usepackage{amsmath, mathrsfs, amssymb}
\usepackage{tikz}

\usepackage{xcolor} % For colors

\usepackage{listings}

\lstdefinestyle{haskellstyle}{
  language=Haskell,
  basicstyle=\ttfamily\tiny,
  keywordstyle=\color{blue},
  commentstyle=\color{green!60!black},
  stringstyle=\color{red},
  numberstyle=\tiny\color{gray},
  numbers=none,
  showstringspaces=false,
  tabsize=2,
  breaklines=true,
  frame=single,
  framerule=0.5pt,
  backgroundcolor=\color{gray!5},
  literate={\$}{{\$}}1, % <--- penting! agar tanda $ tidak bikin error
}


\newcommand\restr[2]{{% we make the whole thing an ordinary symbol
  \left.\kern-\nulldelimiterspace % automatically resize the bar with \right
  #1 % the function
  \littletaller % pretend it's a little taller at normal size
  \right|_{#2} % this is the delimiter
  }}

\newcommand{\littletaller}{\mathchoice{\vphantom{\big|}}{}{}{}}


\title{Permainan Kartu Kabul}
\subtitle{Proyek Pemrograman Fungsional}
\author{Muhammad Fasya Syaifullah}
\date{17 November 2025}
\institute{Universitas Indonesia}

\begin{document}

\begin{frame}[plain,t]
\titlepage
\end{frame}

\begin{frame}{Daftar Isi}
    \tableofcontents
\end{frame}

%=============================================================================================
\section{Kemajuan}
\begin{frame}{Target}
    \textbf{Target Progress 4 (17 November 2025)}
    \begin{itemize}
        \item Membuat fungsi-fungsi \textit{Powerups} dan \textit{trigger Powerup}
        \item \textbf{(maju)} Eksplorasi \textit{user interface}.
    \end{itemize}

    Kemajuan:
    \begin{itemize}
        \item Target tidak tercapai sepenuhnya, ada beberapa bagian yang perlu dimodifikasi sehingga alur permainan lebih natural karena adanya powerup.
        \item Target fungsionalitas belum tercapai.
        \item \textit{User Interface} akan menggunakan Scotty.
    \end{itemize}
\end{frame}

\begin{frame}[fragile]{Contoh Fungsi}
\begin{lstlisting}[style=haskellstyle]
promptIndex :: Int -> IO Int
promptIndex maxIdx = do
    putStrLn $ "Pilih posisi kartu (1 - " ++ show maxIdx ++ "): "
    input <- getLine
    case readMaybe input of
        Just n | n >= 1 && n <= maxIdx -> return n
        _ -> do 
            putStrLn "Input tidak valid, coba lagi"
            promptIndex maxIdx

peekSelf :: Table -> Int -> IO Table
peekSelf table playerIdx = do
    let ps = players table
        p = ps !! playerIdx
        Hand hs = hand p
        n = length hs
    if n == 0 
        then
            return table
        else do 
            mapM_ (\i -> putStrLn $ show i ++ ". [??]") [1..n]

            idx <- promptIndex n
            let chosen = hs !! (idx - 1)
            showCardRS chosen
            return table
\end{lstlisting}
\end{frame}

\begin{frame}[fragile]{Contoh Fungsi}
\begin{lstlisting}[style=haskellstyle]
peekOpponent :: Table -> Int -> IO Table
peekOpponent table playerIdx = do
    let ps         = players table
        numPlayers = length ps
        me         = ps !! playerIdx
        -- sementara: target = pemain berikutnya
        targetIdx  = (playerIdx + 1) `mod` numPlayers
        target     = ps !! targetIdx
        Hand hsOpp = hand target
        n          = length hsOpp

    if n == 0
      then do
        return table
      else do
        mapM_ (\i -> putStrLn $ show i ++ ". [??]") [1..n]

        idx <- promptIndex n
        let chosen = hsOpp !! (idx - 1)
        showCardRS chosen
        return table
\end{lstlisting}
\end{frame}

\begin{frame}{Output Program (PeekSelf)}
    \centering
    \includegraphics[width=0.6\textwidth]{gambarku/switchWithOpponent.PNG}
\end{frame}

\begin{frame}{Output Program (PeekSO)}
    \centering
    \includegraphics[width=0.8\textwidth]{gambarku/peekSO.PNG}
\end{frame}


\section{Daftar Commit}
\begin{frame}{link commit}

\href{https://github.com/mfasyas/cabo-card-functional/commits/add-feature/}{klik di sini}
    
\end{frame}

\section{Hal yang Dipelajari}
\begin{frame}{Game Making}
    \begin{itemize}
        \item Program yang dibuat masih imperatif (setidaknya terlihat seperti itu). Problmnya, bagaimana cara membuatnya lebih fungsional? Apa yang perlu diubah?
        \item Gunakan \textit{Declarative Rule System} untuk setiap aksi di permainan. Jadi, alih-alih menuliskan menuliskan aturan permainan langsung sebagai kode, akan dibuat mesin "ajaib" yang membaca deklarasi aturan-aturan permainan (\textit{drawCard, peek,} \textit{matching} sebagai perintah natural untuk dilakukan.
    \end{itemize}
\end{frame}
%=============================================================================================
\end{document}